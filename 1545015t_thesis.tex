%#BIBTEX pbibtex thesis
%Header
\documentclass[12pt]{jarticle}
\usepackage[top=40truemm, bottom=50truemm, left=35truemm, right=35truemm]{geometry}
\usepackage[dvipdfmx]{graphicx, color}
\usepackage{ascmac}
\usepackage{color}
\usepackage{amsmath}
\usepackage{subfigure}
\usepackage{bm}
\usepackage{here}
\usepackage{tabularx}

\renewcommand{\figurename}{Fig.}
\renewcommand{\tablename}{Table}

\begin{document}
\author{上森 香}
\date{today} 
\thispagestyle{empty}
\begin{center}
\huge{2018年度}\\
\huge{卒業論文}\\
\vspace{4cm}
\LARGE{Yin-Yang-Zhong格子データ向け\\可視化モジュールの開発}\\
\vspace{4cm}
\LARGE{神戸大学工学部情報知能工学科}\\
\vspace{0.45cm}
\LARGE{上森 香}\\
\vspace{1.2cm}
\LARGE{\underline{指導教員 坂本 尚久 講師}}\\
\hspace{31mm}\LARGE{\underline{陰山 \hspace{4mm} 聡 教授}}\\
\vspace{2cm}
\Large{2019年1月11日}
\end{center}

\newpage
\thispagestyle{empty}
\begin{center}
\Large{\underline{Yin-Yang-Zhong格子データ向け可視化モジュールの開発}}\\
\vspace{1.5cm}
\large{上森 香}
\end{center}
\vspace{1cm}
%=============================================
\section{要旨}
%=============================================
本論文では、可視化パイプライン構築をサポートする基盤環境に対して、Yin-Yang-Zhong格子データ向け可視化機能を提供することによって効率の良い可視化映像の作成ができることを示す。
 
 \newpage
% 目次
\thispagestyle{empty}
\tableofcontents

% 本文
\newpage
\setcounter{page}{1}
%=============================================
\section{序論}
%=============================================

\newpage
%=============================================
\section{提案手法}
%=============================================
本章では、Yin-Yang-Zhong格子データ向け可視化機能を提供するにあたっての手法を提案する。
\newpage
%=============================================
\section{結果}
%=============================================

\newpage
%=============================================
\section{考察}
%=============================================

\newpage
%=============================================
\section{まとめ}
%=============================================

\newpage
%=============================================
\section*{謝辞}
\addcontentsline{toc}{section}{謝辞}
%=============================================

\bibliographystyle{junsrt}
\bibliography{bibtex}
\addcontentsline{toc}{section}{参考文献}
\end{document}
